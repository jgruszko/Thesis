
% ========== Chapter 2
 
\chapter {HPGe Detectors}

\section{P-Type Point Contact Germanium Detectors}
It has long been known that reducing the capacitance of High-Purity Germanium (HPGe) detectors would reduce their noise and energy thresholds. This could done by using a small ``point-like" central contact, instead of the deep well used by coaxial detectors. The first attempts to make germanium detectors with point-contact geometries were made in 1989 by Luke et. al \cite{Luke1989}. Though these detectors had much smaller capacitance than coaxial detectors, they suffered from severe charge-trapping effects, degrading the detector resolution. 

The breakthrough improvement that made this geometry useful in 2007 came with the switch from N-type to P-type detectors. i.e., in switching from drifting electrons to drifting electron holes through the crystal \cite{Barbeau2007}. Since the holes are less susceptible to trapping, PPC detectors can achieve resolutions similar to those of coaxial detectors, with electric fields created primarily through careful control of the charge impurity gradient in the bulk of the crystal. Due to their geometry, PPCs have capacitance of about 1~pF, far lower than that of similarly-sized coaxial detectors. This leads to far lower noise than is found in coaxial detectors, and therefore lower thresholds. While PPCs have masses up to 1 kg, the thresholds that can be achieved are comparable to those of small ($\sim1$~g) x-ray detectors \cite{Barbeau2007}. 


\subsection{PPCs and Low-Energy Recoils}
Because of their low thresholds and high masses, PPCs are a promising technology for low-mass WIMP searches and coherent neutrino-nuclear scattering experiments. The MALBEK and CoGeNT experiments are both low-background single-PPC experiments that have set competitive limits on (and in the case of CoGeNT, claimed observation of a signal consistent with) low-mass WIMPs \cite{MALBEK2015}\cite{COGENT2011}. One previous attempt was made to measure neutrino-nuclear scattering with a PPC \cite{BarbeauThesis}, and the COHERENT collaboration is currently considering using the technology for a future experiment at the Spallation Neutron Source \cite{CoherentSnowmassWP}. 

\subsection{Low-Energy Backgrounds in PPC Detectors}
As the tension between the MALBEK and CoGeNT results demonstrates, these measurements rely on a detailed understanding of background events near the detector threshold. Understanding these events, particularly from x-ray and $\alpha$ particle interactions, is the focus of this thesis. $\alpha$ events are also of great concern to the \MJ~ collaboration, as radon deposition on detector surfaces could contribute to backgrounds at the $0\nu\beta\beta$ Q-value. Though these backgrounds have been studied in N-type coaxial detectors \cite{JohnsonThesis2010}, this work is the first attempt to evaluate them in PPCs.