% ===========Conclusion Chapter
\chapter{Conclusions}
\section{Summary of Results}
In this work, we have demonstrated one possible implementation of an algorithm to identify passivated-surface alpha events in \ppc\ HPGe detectors. By studying alpha events directly using the TUBE scanner, we have characterized the response of the entire alpha-sensitive surface of a detector like those used in the \MJ\ \DEM, describing both the charge trapping/loss and delayed charge recovery (DCR) effect on the passivated and p+ contact surfaces. 

This study has allowed us to identify the phenomenon responsible for the DCR effect, near-surface trapping of holes in the semiconductor crystal. This effect has been shown to be highly significant, leading to loss of over 50\% of the incident alpha energy at some locations on the detector surface. It has also been shown to be highly dependent alpha incidence position's radial distance from the p+ contact. Using a combination of the newly-developed DCR pulse-shape parameter and the previously-implemented A/E parameter (or an alternative measure of pulse drift time), alpha events on all sensitive surfaces can be reliably identified with less than 0.2\% sacrifice of \nonubb\ signal events. 

By applying the DCR analysis to the \MJ\ \DEM\ low-background data, backgrounds in the \nonubb\ region-of-interest have been reduced by a factor of 31, improving the expected sensitivity of the experiment by a factor of 3. 

The measurements taken with TUBE have also allowed us to model the expected alpha background spectral shape in the \DEM. Using an analytic approach, we have studied two models describing the distribution of alpha-emitting radioactive contaminants in the experiment-- one in which the entire passivated surface of each detector has been contaminated, and one in which the background events originate in a point source located some distance above the p+ contact of each detector. We have found that the first model does not match the observed spectrum, and that the second model predicts the spectral features observed in the \DEM\ quite accurately. 

In spite of the many simplifying assumptions used in this study, a two-parameter fit of the model spectrum to the data indicates that the plastics used in the \MJ\ detector unit are likely responsible for the alpha background observed. Though further work is needed to confirm that this is the case, the removal and replacement of these contaminated parts in already being considered. 
 
\section{Proposed Improvements to the DCR Analysis}
Several improvements can still be made to further improve the DCR analysis parameter algorithm, and particularly to reduce the uncertainties associated with its bulk acceptance. The two main contributions to the uncertainty of this value are the pulse shape uncertainty $\sigma_{PS}$ and the stability $\sigma_{stab.}$. 

The pulse-shape uncertainty currently appears to be dominated by the effect of charge trapping. Though one implementation of a charge trapping correction has been described in this work, it relies on very precise calibration of a pulse-maximum energy estimator, and has proven impractical to implement in the \DEM. An alternative approach currently being studied by the \MJ\ Collaboration suggests that a direct measure of the trapping on a waveform-by-waveform basis, based on the drift time of the pulse, can provide a far more precise measurement of the degeneracy of the bulk trapping and DCR effects. This effect can subsequently be corrected for, reducing the width of the bulk-event DCR distribution and allowing reduced signal sacrifice without a reduction in alpha-identification power. 

Multi-site events that evade identification by the A vs. E analysis are also contributing to the measured $\sigma_{PS}$. This effect could be reduced by implementing additional single-site event selection, like a waveform-library-based pulse shape analysis.

The stability-related uncertainty also has two main contributions, both of which can be addressed by future improvements to this analysis: changing pole-zero decay constants and changing noise conditions. The first of these can be addressed by implementing a true pole-zero correction in the \MJ\ analysis, rather than the effective correction used here. Such a correction can be re-calculated based on physics events from weekly energy calibrations, or even more frequently using the onboard pulsers. This would eliminate the effect of small instabilities over months of data-taking seen using the current strategy, in which pole-zero decay constants are calculated once per data set for each channel, and re-calculated when major instabilities are observed. 

The latter can be addressed by implementing waveform filtering prior to the DCR parameter evaluation. Since it is high-frequency noise that cases bulk DCR distribution broadening, and the low-frequency components of the waveform that indicate that delayed charge recovery is occurring, the use of notch, wavelet, or bandpass filtering would  limit the impact of noise quite effectively. 

\section{Future Studies of Alpha Backgrounds}
In the near future, studies with the TUBE scanner will attempt to identify the origin of the observed DCR instability observed in the system. By bias- and temperature-cycling the system, we should be able to distinguish the effects of the two leading hypotheses, surface-channel formation and deep trap filling. Another near-term study will focus on taking alpha data with a longer digitization window-- over 100\,$\mu$s instead of the 30\,$\mu$s used in this work. This will allow us to further characterize the rate of charge re-release. Other studies with the TUBE scanner will focus on characterizing the thickness of the charge trapping region associated with the passivated surface, whether by varying the incidence angle of an alpha source or by measuring the energy-dependent efficiency of various low-energy gamma peaks with a multi-line gamma source like $^{133}$Ba. 

Further work should also be done to study the DCR effect in other detector geometries, like the Canberra's BEGe \ppc, and new inverted-coax geometries being considered by the LEGEND Collaboration. To this purpose, a re-analysis of the TUBE BEGe scan data is being considered, and the \MJ\ group at the University of Washington is proceeding in the design and eventual construction of a similar scanner that will allow surface measurements of larger-diameter detectors.

Upcoming efforts by the \MJ\ Collaboration will also focus on incorporating the observed passivated-surface behavior into simulations of alpha backgrounds. This will allow a more accurate background model for the experiment to be constructed, and will lead to a more precise identification of contaminated elements in the \DEM\ than can be provided by the simplified analytical model used in this work.    

\section{The Future of Double-Beta Decay Searches}
The search for \nonubb\ stands at an exciting crossroads. The current generation of experiments, like the \DEM, CUORE, EXO-200, and others, have released or will be releasing new results in the coming few years. These experiments have used a wide range of techniques, whether in their choice of isotope, detection technique, or background reduction strategies. 

To move to the tonne scale, and to achieve the low backgrounds needed to be sensitive to the entire inverted hierarchy range of $m_{\beta\beta}$, future experiments will have to draw on a variety of these techniques. This is the approach taken by the LEGEND Collaboration, which is drawing on the techniques developed by both the GERDA and the \MJ\ Collaborations. 

One strategy that can make a major impact on an experiment's background rejection capability is particle identification, distinguishing beta decay events from alpha and gamma events. Though it is not typically described in those terms, multi-site event rejection in \ppc\ detectors is a sort of gamma particle identification; in this work, we have shown that the DCR analysis can provide a similar tool to identify alpha events with very little signal event sacrifice. These capabilities, as well as the extremely low ROI background rates seen in the current $^{76}$Ge experiments, provide an exciting path forward for the field. 

This work also demonstrates the power that having a complete description of the physics of detectors can have in rare event searches. When studying more common processes, ``fringe cases" like the passivated surface behavior studied here can be disregarded. In studies of exceedingly rare physics, like \nonubb, however, these cases have to be as completely understood as possible, even if the fraction of the detector affected is far less than 1\%. 

In fact, understanding the phenomena behind these anomalous behaviors can actually serve as a powerful tool, as it has in the case studied here. In my mind, this is an important lesson to retain in developing new low-background techniques. This work and other efforts by the \MJ\ Collaboration have shown that far more information still remains to be extracted from our HPGe signals; by leveraging it, we can reach new levels of sensitivity even with the same, currently-operating experiment. A similar campaign of simulations and relatively simply-accomplished dedicated measurements could lead to further improvements in other experiments as well. 



