\prelimpages
 
%
% ----- copyright and title pages
%
\Title{Surface Alpha Interactions in P-Type Point-Contact HPGe Detectors: Maximizing Sensitivity of $^{76}$Ge Neutrinoless Double-Beta Decay Searches}
\Author{Julieta Gruszko}
\Year{2017}
\Program{UW Physics}

\Chair{Jason Detwiler}{Assistant Professor}{Physics}
\Signature{Ann Nelson}
\Signature{Grey Rybka}

\copyrightpage

 \titlepage  

%
% ----- signature and quoteslip are gone
%

%
% ----- abstract
%


\setcounter{page}{-1}
\abstract{
Though the existence of neutrino oscillations proves that neutrinos must have non-zero mass, Beyond-the-Standard-Model physics is needed to explain the origins of that mass. One intriguing possibility is that neutrinos are Majorana particles, i.e., they are their own anti-particles. Such a mechanism could naturally explain the observed smallness of the neutrino masses, and would have consequences that go far beyond neutrino physics, with implications for Grand Unification and leptogenesis. 

If neutrinos are Majorana particles, they could undergo neutrinoless double-beta decay (\nonubb), a hypothesized rare decay in which two antineutrinos annihilate one another. This process, if it exists, would be exceedingly rare, with a half-life over $10^{25}$ years. Therefore, searching for it requires experiments with extremely low background rates. One promising technique in the search for \nonubb\ is the use of P-type point-contact (\ppc) high-purity Germanium (HPGe) detectors enriched in $^{76}$Ge, operated in large low-background arrays. This approach is used, with some key differences, by the \MJ\ and GERDA Collaborations. 

A problematic background in such large granular detector arrays is posed by alpha particles incident on the surfaces of the detectors, often caused by $^{222}$Rn contamination of parts or of the detectors themselves. In the \textsc{Majorana Demonstrator}, events have been observed that are consistent with energy-degraded alphas originating near the passivated surface of the detectors, leading to a potential background contribution in the region-of-interest for neutrinoless double-beta decay. However, it is also observed that when energy deposition occurs very close to the passivated surface, high charge trapping occurs along with subsequent slow charge re-release. This leads to both a reduced prompt signal and a measurable change in slope of the tail of a recorded pulse. Here we discuss the characteristics of these events and the development of a filter that can identify the occurrence of this delayed charge recovery (DCR) effect, allowing for the efficient rejection of passivated surface alpha events in analysis. 

Using a dedicated test-stand called the TUM Upside-down BEGe (TUBE) scanner, we have characterized the response of a \ppc\ detector like those used in the \DEM\ to alphas incident on the sensitive surfaces, developing a model for the radial dependence of the energy loss to charge trapping and determining the dominant mechanism behind the delayed charge effect. We have also used these measurements to demonstrate the complementarity of the DCR analysis with the drift-time analysis that is used to identify alpha background candidate events in the GERDA detectors. Using these two methods, we demonstrate the ability to effectively reject all alpha events (to within statistical uncertainty) with only 0.2\% bulk event sacrifice. 

Applying the DCR analysis to the events observed in the \MJ\ \DEM, we find that it reduces the backgrounds in the \nonubb\ region-of-interest by a factor of 31, increasing the expected experimental sensitivity by a factor of 3 over the lifetime of the \DEM. The results of the dedicated measurements in the TUBE scanner can be used to build a background model for alpha decays in the \DEM ; here, we examine two simplified geometric models for the alpha source distribution and find that the observed spectral shape is consistent with alpha events originating in the plastics of the detector units. 
 }
%
% ----- contents & etc.
%
\tableofcontents
\listoffigures
\listoftables  % I have no tables
 
%
% ----- glossary 
%
\chapter*{Glossary}      % starred form omits the `chapter x'
\addcontentsline{toc}{chapter}{Glossary}
\thispagestyle{plain}
%
\begin{glossary}
\item[MJD] The \MJ\ \MJDemo\ 
\item[HPGe] High-purity Germanium
\item[PPC] P-type point contact detector.
\end{glossary}
